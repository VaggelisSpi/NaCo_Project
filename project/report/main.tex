\documentclass{article}
\usepackage{graphicx} % Required for inserting images
\usepackage{biblatex}
% \usepackage[sortcites=true,sorting=nyt,backend=biber]{biblatex}
\usepackage{hyperref}
\usepackage{amsmath}
\usepackage{amssymb}
\usepackage{caption}
\usepackage{subcaption}

\bibliography{references}

\title{Evolving Training Sets for Peptide Discrimination via Evolutionary Algorithms}
\author{Loek Gerrits (s1032343), Evangelos Spithas (s1125593), Bart van Nimwegen}
\date{\today}

\begin{document} 

\maketitle

\section{Introduction}
In human bodies, the immune system is responsible for detecting pathogens and exterminating them. 
One of the tools at its disposal, is the T cells. T cells grow in the thymus, and each one is responsible for
identifying and attacking specific cells. However, that means that it is possible for healthy cells to be
attacked as well. In order to mitigate this, human cells are presented to the T cells, and the T cells that attack them 
are eliminated. This process is called Negative Selection (NS). In practice, the amount of possible cells is very large, and as a result only a sub-set of self peptides are presented
in the T cells. 

Artificial Immune Systems (AIS) are systems that draw inspiration from the human immune system, similarly to how Neural 
Networks (NNs) are inspired by the human nervous system. In this project, we would like to explore how we can find 
optimal subsets to train an NS algorithm for peptide selection. We will do this using 3 different methods, randomly 
sampling a subset, using a greedy algorithm and an evolutionary algorithm to produce them. Afterwards, we will train the 
negative selection algorithm with each one of them, and evaluate its performance against different sets of harmful 
peptides, such as HIV and ebola cells. 



\section{Methods}

\subsection{Datasets}

\subsubsection{Random}


\subsubsection{Greedy Algorithm}


\subsubsection{Evolutionary Algorithm}

\paragraph{AA composition}

\paragraph{AA frequency}

\paragraph{Exchangeability}


\subsection{Experiment}


\subsection{Analysis}


\section{Results}



\section{Discussion}



\section{Conclusion}

The end.

\printbibliography

\appendix

\end{document}